\documentclass[11pt]{article}
\usepackage{graphicx}
\usepackage{fixltx2e}
\graphicspath{ {images/} }


\title{Literary Review}
\author{Michiel Willems}


\begin{document}
	
\section{Title Page (1p)}
\begin{itemize}
	\item Title: Security in automobiles: OBD-II Access Control.
\end{itemize}

\section{Abstract (1/2p)}

\section{Introduction (3-5 pages)}
\begin{itemize}
	\item \textbf{Problem Statement + Motivation}: briefly explain (through examples) the vulnerability of the current modern automobile (attack vectors).
	\item \textbf{Challenges}: basically sufficiently secure + fast.
	\item \textbf{Context}: How does my solution fit into other existing security solutions (e.g. Authentication of nodes, secure key storage, etc).
	\item \textbf{Proposed contribution}: Secure OBD-II authentication.
	\item \textbf{Outline} (basically this text).
\end{itemize}

\section{Literature Review(17-22p)}

\subsection{Vehicle network Infrastructure (2p)}
\begin{itemize}
	\item ECU's.
	\item Communication protocols (can etc).
	\item Gateway.
\end{itemize}

\subsection{OBDII protocol (2-3p)}
\begin{itemize}
	\item The goal of OBDII.
	\item brief history.
	\item DLC.
	\item PID's.
	\item Scan Tools
	\item Signalling protocols (can etc).
\end{itemize}

\subsection{CAN Protocol (2-3p)}
\begin{itemize}
	\item brief history.
	\item Architecture.
	\item CAN frames.
	\item CAN priorities.
	\item CAN security vulnerabilities.
\end{itemize}

\subsection{OBDII Security Threats (4-5p)}
\begin{itemize}
	\item Architectural shortcomings.
	\item examples of potential attacks.
	\item examples of impact of attacks.
	\item e.g. charlie miller and chris valasek.
\end{itemize}

\subsection{Related Work (8-10p)}
\begin{itemize}
	\item Survey of potential attack vectors.
	\item other proposed OBD security measure (e.g Seed-key).
	\item Internal vehicle security (e.g. Leia, VatiCAN, VulCAN, CANcrypt).
	\item External vehicle security (e.g. DoIp, UDS, Connected Repair shop, etc).
	\item Automobile security requirements/assurance levels: Autosar, ASIL, ASEAL, ENISA. 
\end{itemize}

\subsection{Conclusion (1p)}
\begin{itemize}
	\item propose that is is proven that proposed OBD access control solution is a solid addition to the existing automotive security landscape.
\end{itemize}

\section{Preliminaries}
\subsection{Hardware requirements/limitations (2-4p)}
\begin{itemize}
	\item Typical ECU hardware (e.g 8/16 bit microcontroller).
\end{itemize}

\subsection{Security Primitives (5-6p)}
\begin{itemize}
	\item ECC/ECDSA/ECSS
	\item Hashing (sha)
	\item Hmac-Sha 
	\item Secure API for key operations (SGX enclave?)
\end{itemize}

\section{Solution}
\subsection{Role based Access Control (8-10p)}
\begin{itemize}
	\item types of roles: repairman, dealer, police, etc.
	\item allocation of public keys to roles.
	\item Authentication algorithm (both scenarios).
	\item Key distribution and access.
\end{itemize}

\section{Implementation (8-10p)}
\begin{itemize}
	\item Hardware overview (AT90CAN boards with CAN controller).
	\item Implementation assumptions: RNG, OBD-Messages, other Gateway functionality, etc. 
	\item implementation specifics (permission table, private key API, etc).
	\item Demo results (pictures).
	\item used libraries: AVR-crypto, tinyECC, SANCUS.
\end{itemize}

\section{Evaluation (5p)}
\subsection{Security Evaluation}
\begin{itemize}
	\item cryptographic strength: ECC, Shared secret, Hmac.
	\item security against potential attacks: MiM, Side-channel, bruto force, etc.
	\item importance of secure private key storage.
\end{itemize}

\subsection{Performance Evalution}

\section{Future Work (3-4p)}
\begin{itemize}
	\item study on which/how many roles are required.
	\item Private Key infrastructure.
	\item Raising Automotive industry awareness.
\end{itemize}

\section{Conclusion (1-2 p)}

\section{References}.






\end{document}

