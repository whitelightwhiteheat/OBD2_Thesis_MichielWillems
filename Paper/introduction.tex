\chapter{Introduction}
\label{sec:introduction}

\section{Context}
\label{sec:context}

The automotive industry has evolved greatly since the introduction of the Ford Model T in 1917. Although the main purpose of these machines remains the same (e.g. getting someone from point A to B swiftly), their relative comfort, speed, safety, and efficiency has improved dramatically. Primarily due to the introduction of electronic computers into the vehicle's architecture. The modern vehicle has been appropriately called a "Computer on wheels" in \cite{Klinedinst05}, since each one contains up to 100 millions lines of code, spread out over tens of Electronic Control Units (ECUs) \cite{Pike15}. Each ECU is an embedded computer that is designed to perform a specific function (e.g. braking, opening the door, speed control, etc.). In addition to having this wide variety of embedded devices, a modern vehicle will also employ a data bus that allows all ECU's inside a vehicle to communicate with each other. There are multiple standards that are employed even within a single vehicle, but the CAN (Controller Area Network) protocol is the most widely used one \cite{VatiCAN}, Hence we focus on it in this paper. \\ \\ Alongside internal communication networks many modern models of vehicles also support some way of performing external communications. This can range from vehicle-to-infrastructure (V2I) (e.g. wireless gas payment at a gas station, wireless diagnostics at a repair shop or even virtual traffic lights), vehicle-to-vehicle communications (V2V) (e.g. automatically following another vehicle), vehicle-to-network (V2N) (connecting your vehicle to an already existing network, like the cellular communications network for example) and vehicle-to-pedestrian (V2P) \cite{Kleberger15,Russel17,Ahmed}. This extended functionality greatly improves the vehicle's flexibility, comfort and efficiency. However, this also makes them increasingly vulnerable to a wide variety of cyber attacks. These attacks can be mounted via the various interfaces that can communicate with the external world. This is exemplified by car thieves abusing remote keyless entry (RKE) systems to gain access to a car \cite{KeeLoq,MillerA}, remotely causing a vehicle to think it is having a tire problem by interfering with the tire pressure monitoring system (TPMS) \cite{MillerA} or even compromising a vehicle through the Bluetooth interface \cite{Kosher2,Kosher}. \\ \\
there are numerous points of entry to the internal vehicle network, both physical (Breaking into the vehicle and directly connecting to the network) and remote (Bluetooth, TPMS or Tire Pressure Monitoring System, Radio system, etc.) \cite{MillerA}. Take Bluetooth for example: many cars include Bluetooth functionality to allow users to connect their phones and play music. Bluetooth has a large protocol-stack and it has been shown by \cite{Bluetooth} that it's design possesses some serious security flaws. Discoverability, bluejacking, bluesnarfing and backdoor attacks are just a couple of examples that exploit these flaws. By exploiting the vulnerabilities of a car's Bluetooth interface, a malicious agent is able to interfere with the internal network remotely (using his/her mobile phone). This problem is compounded by the fact that it is easy for a phone to get compromised (e.g. by visiting a malicious website) \cite{Yadav16}. This problem would be solved by using a more secure version that does not contain the aforementioned vulnerabilities.



\section{Motivation}
\label{sec:motivation}

The On-board Diagnostics (OBD-II) port is one of the potential attack vectors.  OBD-II systems are widely deployed in auto-mobiles as a way of getting diagnostics information from the vehicle. OBD-II introduces a physical interface inside the vehicle passanger compartment (usually under the steering wheel) called the Data Link Connector (DLC). This physical interface allows full access to the internal network. It has been shown in \cite{MillerA,Yadav16,MillerB,MillerC} that a set of messages or signals can be injected on a car's CAN bus to control key components (e.g. lights, locks, brakes, and engine) as well as injecting code into ECUs. A couple of security solutions were proposed that are designed to amend this problem, like the seed-key algorithm discussed in \cite{Yadav16}. However, it is our opinion that a comprehensive solution to this problem is yet to be proposed. The focus of this thesis will be to try and mitigate the vulnerability of OBD-II, by introducing access control to the OBD-II interface. 

 

\section{Challenges} \label{sec:challenges}
The main challenge of this research topic is to introduce a solution that ports well to the kind of hardware that is found in vehicles. Introducing new components into the internal vehicle network would surely simplify things. If this were the case, the solution could consist of introducing a small component that acts as a firewall for the OBD-II interface. However this implies that any potential real-world implementation requires the installation of this component into millions of currently in-use vehicles. Which, being a very costly endeavour, would deter any manufacturers from doing so. Therefore, a software-based approach is preferable. It is easy to deploy such a solution on extant cars, without requiring hardware modifications or excessive expenses. However, This approach introduces it's own challenges; namely, the limitations of ECU micro controllers. Indeed, any solution that isn't portable to a typical vehicle network because of memory limitations, limited processing power, incompatible architectures, etc., is ultimately rendered useless. It is worth noting that the solution proposed here is not intended to (and will not) protect against attacks using other attack vectors (e.g. TPMS, Bluetooth, etc.). This also applies to physical attacks. Indeed, any attacker gaining physical access to the vehicle has to ability to directly interface with the vehicle network (e.g. by physically tapping into the CAN bus). Typically only the owner of the vehicle has this privilege, and it is safe to assume this person is reluctant to compromise the safety of their own vehicle. unauthorised physical access should be mitigated by different means (e.g. car alarms, safe RKE systems, the authorities, etc.). The main challenges of this thesis paper are summarized as follows:

\begin{itemize}
	\item \textbf{portability:} The solution should port well to existing vehicle networks. 
	\item \textbf{Security:} The solution should be sufficiently secure according to current computer safety standards. 
	\item \textbf{Speed:} The solution should not impede the operation of other processes running on the same network.
\end{itemize}

\section{Contributions}
\label{sec:contributions}

The contributions of this thesis can be summarized as follows:

\begin{itemize}
	\item Overcoming the security limitation identified by the unauthorised access to the vehicle CAN network via the OBD-II port by designing and developing a role-based access control model based on public key cryptography.
	
	\item Providing an open source proof of concept implementation of our designed model on a CAN-enabled resource-constrained ECU that is used in various automotive models. Furthermore, we evaluate and show that our approach is secure, feasible and lightweight in terms of memory footprint and runtime overhead.
\end{itemize} 

\section{Text Outline}
The remainder of this text is separated into 8 chapters:
\begin{itemize}
	\item \textbf{Chapter \ref{chap:background}} is concerned with providing some insight in the operation of intra vehicle networks. In lieu with this intention, the CAN protocol and the OBD-II standard are discussed in detail. The main takeaway of this chapter consists of a solid technical background, allowing the reader to correctly interpret the rest of this paper.
	
	\item \textbf{Chapter \ref{chap:related_work}} is a survey of several studies that are topically related to this paper. The idea is that this will help to contextually situate our study, as well as getting the reader up to date with the current state of affairs vis-\`a-vis automotive network security.
	
	\item \textbf{Chapter \ref{chap:problem_statement}} illustrates the main research topic of this paper. Namely, the vulnerability of the OBD-II interface. A detailed discussion of this problem is held, illustrated by some example attacks. A suitable attacker model is also defined. The insights made in this chapter will give us a set of security requirements that any proposed solution should meet.
	
	\item \textbf{Chapter \ref{chap:preliminaries}} serves as a technical primer to all the security systems that are applied in our proposed solution. These preliminaries are included for readers without a software security background, in an effort to help them understand how these systems work, and why they were chosen.
	
	\item \textbf{Chapter \ref{chap:solution}} is devoted to our proposed solution, namely OBD-II role-based access control. A detailed description of the design is given, accompanied by a series of diagrams.
	
	\item \textbf{Chapter \ref{chap:implementation}} documents how the proposed
	
	 solution from chapter \ref{chap:solution} was implemented in hardware. 
	
	\item \textbf{Chapter TODO}
	
	\item \textbf{Chapter TODO}
\end{itemize}